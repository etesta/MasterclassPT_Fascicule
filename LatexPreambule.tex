\usepackage{helvet}
\renewcommand{\familydefault}{\sfdefault}

\usepackage{tocloft}
\usepackage{caption}
\usepackage{alltt}
\usepackage[dvipsnames,table]{xcolor}
\usepackage[french]{babel}
\usepackage[latin1]{inputenc}
\usepackage{amsmath,amsbsy,amssymb,empheq}
% \newtheorem{theo}{Th�or�me}   % Environnement theoreme
% \usepackage{amsfonts}
\usepackage{graphicx}
\usepackage{listings,comment}
\usepackage{tabularx} % Pour avoir des tableaux d'une largeur bien d�finie
\usepackage{multirow} % pour faire fusionner des lignes dans des tableaux
\usepackage[babel=true]{csquotes} % csquotes va utiliser la langue d�finie dans babel
\usepackage[rightcaption]{sidecap}% Pour utiliser SCfigure (figure encadree)
\usepackage[squaren,Gray]{SIunits} % pour avoir de jolies unit�s
\usepackage{enumitem}
\usepackage{hyperref}
\usepackage{color}
\usepackage{xifthen}
\usepackage[makeroom]{cancel}
\usepackage{csquotes}
\usepackage{subfig} % Pour avoir des sous-figures
\usepackage{bigints}
\usepackage{float}
\usepackage{icomma} 
\usepackage{textcomp} 

\usepackage{sectsty}
%\allsectionsfont{\color{blue}\itshape\underline}
% \sectionfont{\color{red}\itshape\selectfont}
\sectionfont{\color{red}}
\subsectionfont{\color{blue}}


\everymath{\displaystyle}



% Format des pages
%\pagestyle{headings}  
\oddsidemargin  -1cm
\evensidemargin -1cm
\textwidth 17cm % A ajuster pour les commentaires dans la marge
% \textwidth 18cm
\topmargin -2cm
\textheight 26cm
\makeatletter
\@addtoreset{chapter}{part}
\makeatother

\makeatletter
\newcommand*{\toccontents}{\@starttoc{toc}}
\makeatother


% -------------------------------------------------------------------------------

\newenvironment{etapes}{%
  \begin{itemize}
  \def\etape{\item[\textbullet] }
}{
  \end{itemize}
}

% fonction rectangle
\newcommand{\sinc}[0]{\mathrm{sinc}}

% quotation marks
\newcommand{\g}[1]{\og#1\fg}

% Rightarrow
\newcommand{\Ra}[0]{\Rightarrow}

\newcommand{\ora}[1]{\overrightarrow{#1}} 

\newcommand{\er}{\ensuremath{\va*{e}_{r}}}
\newcommand{\etheta}{\ensuremath{\va*{e}_{\theta}}}
\newcommand{\ephi}{\ensuremath{\va*{e}_{\varphi}}}
\newcommand{\vr}{\ensuremath{\va{r}}}

\newcommand{\ex}{\ensuremath{\va*{e}_{x}}}
\newcommand{\ey}{\ensuremath{\va*{e}_{y}}}
\newcommand{\ez}{\ensuremath{\va*{e}_{z}}}


% \noyau{C}{12}{6} pour le carbone
\newcommand{\noyau}[3]{\prescript{#2}{#3}{\mathrm{#1}}} 

\newcommand{\id}[1]{\mathrm{#1}}  

\newcommand{\mr}[2]{\multirow{#1}{*}{#2}}  
\newcommand{\mc}[3]{\multicolumn{#1}{#2}{#3}}  

% texte en couleur
\newcommand{\tc}[2]{\textcolor{#1}{#2}}
\newcommand{\tcb}[1]{\textcolor{blue}{#1}}
\newcommand{\tcr}[1]{\textcolor{red}{#1}}

\newcommand{\al}[0]{
\setlength{\abovedisplayskip}{0pt}
\setlength{\belowdisplayskip}{0pt}    
}

\newcounter{otheryear}
\setcounter{otheryear}{\the\year}
\ifthenelse{\cnttest{\month}{>}{6}}{\addtocounter{otheryear}{1}}{\addtocounter{otheryear}{-1}}
\def\semestre{\ifthenelse{\cnttest{\month}{>}{6}}{\the\year -\theotheryear}{\theotheryear -\the\year}}
\graphicspath{{Figures/}}

\usepackage{fancyhdr}
\fancypagestyle{thetitlepage}{
\fancyhead{} % clear all header fields
\fancyfoot{} % clear all footer fields
\renewcommand{\headrulewidth}{0pt}
\renewcommand{\footrulewidth}{0pt}
}



% Environnement ``reponse'' et questions
\usepackage{framed} % Pour encadrer du texte
\setlength{\FrameRule}{\fboxrule}
\setlength{\FrameSep}{3\fboxsep}

%\patchcmd\chapter{\null\vfil}{}{}{}
% Environnement ``question'' recupere du package ``exam.sty''
% compteur pour num�roter les questions � la suite dans un m�me TP.
\newcounter{numQuestion}
\newcommand{\resetQuestions}{\setcounter{numQuestion}{1}}
\resetQuestions 

\newenvironment{questions}[1][start=\thenumQuestion]%
{%
  \setlength{\leftmargini}{6pt}
  \setlength{\leftmarginii}{8mm}
  \setlength{\leftmarginiii}{10mm}
  \renewcommand{\labelenumi}{{\bf \theenumi.}}
  \renewcommand{\labelenumiii}{(\theenumiii)}

  % Cadre autour de l'environnement ``questions''
  % \begin{framed}\begin{enumerate}[#1]\addtolength{\itemsep}{2mm}
  % \def\question{\item}\let\ques\question}
  % {\end{enumerate}\end{framed}}
  % Pas de cadre autour de l'environnement ``questions''
  \begin{enumerate}[#1]\addtolength{\itemsep}{2mm}
  \def\question{\item}\let\ques\question
}%
{%
  \setcounter{numQuestion}{\theenumi}
  \addtocounter{numQuestion}{1}
  \end{enumerate}
}

\newenvironment{reponse}
{ %\leftskip1.5em
  \begin{framed}\color{blue}%
  }{%
%  \vspace{-2pt}% ajustement d'espacement en bas  
  \color{black}\end{framed}%
}

\newenvironment{reponsefigure} 
{ %\leftskip1.5em   
  \begin{figure}%
  }{% %  \vspace{-2pt}
  % ajustement d'espacement en bas    
  \end{figure}% 
}

\newenvironment{reponseSCfigure} 
{ %\leftskip1.5em
  \begin{SCfigure}%
  }{%
%  \vspace{-2pt}% ajustement d'espacement en bas  
  \end{SCfigure}%
}
